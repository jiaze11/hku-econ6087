\documentclass[aspectratio=169]{beamer}
\usetheme{metropolis}
\setbeamertemplate{frame numbering}[fraction]

\usepackage{graphicx}
\usepackage{outlines}

\title{Introduction}
\author{Jiaze Li}
\institute{The University of Hong Kong}
\date{\today}

\titlegraphic{\hfill\includegraphics[height=0.5in]{assets/hku.jpg}}

\begin{document}

\maketitle

\begin{frame}{Overview}
    \tableofcontents
\end{frame}

\section{Python}

\begin{frame}{Python}
    \begin{columns}
        \begin{column}{0.8\textwidth}
            \begin{outline}
                \1<1-> In this course, we will use \textbf{Python} for textual analysis.
                \vspace{3ex}
                \1<2-> Why?
                    \2<3-> Python has \alert{rich libraries} for textual analysis.
                    \2<4-> Python has \alert{clear syntax}.
                    \2<5-> \alert{AI agents} are good at Python.
            \end{outline}
        \end{column}
        \begin{column}{0.2\textwidth}
            \centering
            \includegraphics[width=\textwidth]{assets/python.png}
        \end{column}
    \end{columns}
\end{frame}

\begin{frame}{Installation}
    \begin{columns}
        \begin{column}{0.8\textwidth}
            \begin{outline}
                \1<1-> The best way to install Python is \textbf{NOT} to install Python.
                \vspace{3ex}
                \1<2-> The standard way to install Python is through \texttt{conda}, an open-source package and environment manager.
                    \2<3-> I recommend \texttt{conda-forge}, a community-led channel. See \url{https://conda-forge.org/}.
                    \2<4-> Alternatively, you can install \texttt{Anaconda}, a commercial distribution. See \url{https://www.anaconda.com/}.
                \vspace{3ex}
                \1<5-> If you need to use advanced modules like \texttt{torch}, consider installing Python through \texttt{uv}, a fast package manager. See \url{https://docs.astral.sh/uv/}.
            \end{outline}
        \end{column}
        \begin{column}{0.2\textwidth}
            \centering
            \includegraphics<2>[width=\textwidth]{assets/conda.png}
            \includegraphics<3>[width=\textwidth]{assets/conda-forge.png}
            \includegraphics<4>[width=\textwidth]{assets/anaconda.png}
            \includegraphics<5>[width=\textwidth]{assets/uv.png}
        \end{column}
    \end{columns}
\end{frame}

\begin{frame}{Installation}
    \begin{outline}
        \1<1-> 
    \end{outline}
\end{frame}

\end{document}