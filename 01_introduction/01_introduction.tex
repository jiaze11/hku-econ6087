\documentclass[10pt, aspectratio=169]{beamer}
\usetheme{metropolis}
\setbeamertemplate{frame numbering}[fraction]
\setbeamercovered{transparent}

\usepackage{ctex}
\usepackage{fontspec}
\usepackage{graphicx}
\usepackage{listings}
\usepackage{outlines}
\usepackage{xcolor}
\usepackage{tcolorbox}

\definecolor{codegreen}{rgb}{0,0.6,0}
\definecolor{codegray}{rgb}{0.5,0.5,0.5}
\definecolor{codepurple}{rgb}{0.58,0,0.82}
\definecolor{backcolour}{rgb}{0.95,0.95,0.92}
\definecolor{deepblue}{rgb}{0,0,0.5}

\lstset{
    language=Python,
    backgroundcolor=\color{backcolour},
    commentstyle=\color{codegreen},
    keywordstyle=\color{deepblue},
    numberstyle=\tiny\color{codegray},
    stringstyle=\color{codepurple},
    basicstyle=\ttfamily\small,
    breakatwhitespace=false,         
    breaklines=true,                 
    keepspaces=true,                 
    showstringspaces=false,
    tabsize=2,
    frame=single,
    rulecolor=\color{gray!30}
}

\title{Introduction}
\author{Jiaze Li}
\institute{The University of Hong Kong}
\date{February 11, 2026}

\titlegraphic{\hfill\includegraphics[height=0.5in]{assets/hku.jpg}}

\begin{document}

\maketitle

\begin{frame}{Overview}
    \tableofcontents
\end{frame}

\section{Python Installation}

\begin{frame}{Python}
    \begin{columns}
        \begin{column}{0.8\textwidth}
            \begin{outline}
                \1<1-> In this course, we will use \textbf{Python} for textual analysis.
                \vspace{3ex}
                \1<2-> Why?
                    \2<3-> Python has \alert{rich libraries} for textual analysis.
                    \2<4-> Python has \alert{clear syntax}.
                    \2<5-> \alert{AI agents} are good at Python.
            \end{outline}
        \end{column}
        \begin{column}{0.2\textwidth}
            \centering
            \includegraphics[width=\textwidth]{assets/python.png}
        \end{column}
    \end{columns}
\end{frame}

\begin{frame}{Installation}
    \begin{columns}
        \begin{column}{0.8\textwidth}
            \begin{outline}
                \1<1> The best way to install Python is \textbf{NOT} to install Python.
                \vspace{3ex}
                \1<2-3> The standard way to install Python is through \texttt{conda}, an open-source package and environment manager.
                    \2<3> I recommend \texttt{conda-forge}, a community-led channel. See \url{https://conda-forge.org/}.
                \vspace{3ex}
                \1<4-> If you need to use advanced modules like \texttt{torch}, consider installing Python through \texttt{uv}, a fast package manager. See \url{https://docs.astral.sh/uv/}.
            \end{outline}
        \end{column}
        \begin{column}{0.2\textwidth}
            \centering
            \includegraphics<2>[width=\textwidth]{assets/conda.png}
            \includegraphics<3>[width=\textwidth]{assets/conda-forge.png}
            \includegraphics<4>[width=\textwidth]{assets/uv.png}
        \end{column}
    \end{columns}
\end{frame}

\begin{frame}{Startup}
    \begin{outline}
        \1<1-> You can check if you have \texttt{conda-forge} installed correctly by running \texttt{conda --version} in your \alert{Miniforge Prompt} (Windows) or \alert{Terminal} (macOS/Linux).
            \2 For Windows users, if you want to use \texttt{conda} in Terminal, you should run \texttt{conda init} in Miniforge Prompt once.
        \vspace{3ex}
        \1<2-> Your terminal should look like these:
            \2 \texttt{(base) C:\textbackslash Users\textbackslash Jiaze Li>}
            \2 \texttt{(base) jiaze@jiaze-legion:\~{}\$}
    \end{outline}
\end{frame}

\begin{frame}{Environments}
    \begin{outline}
        \1<1-> \texttt{conda} is a package and environment manager.
            \2<2-> Packages are collections of code that provide specific functionality.
                \3 Examples: \texttt{pandas}, \texttt{matplotlib}, etc.
            \2<3-> Environments are isolated spaces where you can install packages without affecting other environments.
        \vspace{3ex}
        \1<4-> To create a new environment, run \texttt{conda create -n <env\_name> python=<version>} in your \alert{Terminal}.
            \2 For example, \texttt{conda create -n py312 python=3.12} will create a new environment named \texttt{py312} with Python 3.12 installed.
        \vspace{3ex}
        \1<5-> For many commands (not only \texttt{conda}), you may see \texttt{Proceed ([y]/n)?}. If you know what you are doing, type \texttt{y} and press \texttt{Enter}.
        
    \end{outline}
\end{frame}

\begin{frame}{Environments (Cont.)}
    \begin{outline}
        \1<1-> To list all your environments, run \texttt{conda env list}.
            \2 \texttt{base} is the default environment that comes with \texttt{conda}. You should avoid installing packages in the \texttt{base} environment to prevent conflicts.
        \vspace{3ex}
        \1<2-> To activate the environment, run \texttt{conda activate <env\_name>}.
            \2 Every time you run commands in the terminal, you should make sure you are in the correct environment.
            \2 Once you activate the environment, your terminal should look like these:
                \3 \texttt{(py312) C:\textbackslash Users\textbackslash Jiaze Li>}
                \3 \texttt{(py312) jiaze@jiaze-legion:\~{}\$}
        \vspace{3ex}
        \1<3-> For more commands, run \texttt{conda --help} or check the official documentation at \url{https://docs.conda.io/}.
    \end{outline}
\end{frame}

\begin{frame}{Packages}
    \begin{outline}
        \1<1-> To install packages in the activated environment, run \texttt{conda install <package\_name\_1> <package\_name\_2> ...}.
            \2 For example, \texttt{conda install jupyterlab pandas} will install the \texttt{jupyterlab} and \texttt{pandas} packages in the currently activated environment.
        \1<2-> Not all packages are available on \texttt{conda-forge}. You should check the official documentation of the package for installation instructions.
            \2 For example, \texttt{torch} no longer supports installation through \texttt{conda} since version 2.6.0.
        \1<3-> Recommended packages for general use:
            \2 \texttt{jupyterlab}: Add support for .ipynb files, which are interactive notebooks that allow you to run code and see the output in the same document.
            \2 \texttt{pandas}: A powerful library for data manipulation and analysis
    \end{outline}
\end{frame}

\section{IDE}

\begin{frame}{IDE}
    \begin{columns}
        \begin{column}{0.8\textwidth}
            \begin{outline}
                \1<1-> In principle, you can write any code in any text editor like Notepad, but an Integrated Development Environment (IDE) can provide useful features like syntax highlighting, code completion, debugging, etc.
                \1<2> I recommend using \alert{Visual Studio Code}, an open-source code editor developed by Microsoft. See \url{https://code.visualstudio.com/}.
                \1<3> You may also consider \alert{Cursor}, a code editor based on VS Code that integrates AI features. See \url{https://www.cursor.com/}.
                    \2 Caveat: Although Cursor is forked from VS Code, some extensions like Data Wrangler may not work properly in Cursor.
            \end{outline}
        \end{column}
        \begin{column}{0.2\textwidth}
            \centering
            \includegraphics<2>[width=\textwidth]{assets/vsc.png}
            \includegraphics<3>[width=\textwidth]{assets/cursor.png}
        \end{column}
    \end{columns}
\end{frame}

\begin{frame}{Extensions}
    \begin{outline}
        \1<1-> Extensions for Python development:
            \2 \alert{Python}: Provides rich support for Python development. See \url{https://marketplace.visualstudio.com/items?itemName=ms-python.python}.
            \2 \alert{Jupyter}: Provides support for Jupyter notebooks. See \url{https://marketplace.visualstudio.com/items?itemName=ms-toolsai.jupyter}.
        \1<2-> Recommended extensions for Python development:
            \2 \alert{GitHub Copilot Chat}: Provides AI-powered code suggestions and explanations. See \url{https://marketplace.visualstudio.com/items?itemName=GitHub.copilot-chat}.
            \2 \alert{Data Wrangler}: Provides a visual interface for data manipulation and analysis. See \url{https://marketplace.visualstudio.com/items?itemName=ms-toolsai.datawrangler}.
    \end{outline}
\end{frame}

\begin{frame}
    \begin{figure}
        \centering
        \includegraphics[width=\textwidth]{assets/workspace.png}
    \end{figure}
\end{frame}

\section{Python}

\begin{frame}[fragile]{Running Your First Code}
    \begin{outline}
        \1<1-> Create a new file and save it with the \texttt{.ipynb} extension (Jupyter Notebook).
        \vspace{3ex}
        \1<2-> At the \alert{upper-right corner} of the VS Code editor, click \alert{"Select Kernel"} and choose the \texttt{conda} environment we created.
        \vspace{3ex}
        \1<3-> Type the following in a code cell and run it.
        \begin{lstlisting}[language=Python]
print("Hello, World!")
        \end{lstlisting}
    \end{outline}
\end{frame}

\begin{frame}[fragile]{Function}
    \begin{outline}
        \1<1-> The \texttt{print()} function is technically defined as follows:
        \begin{lstlisting}[language=Python, basicstyle=\ttfamily\footnotesize]
def print(
    *values: object,
    sep: str | None = " ",
    end: str | None = "\n",
    file: None = None,
    flush: bool = False
) -> None: ...
        \end{lstlisting}
        \1<2-> You can move your mouse cursor over the function name to see its definition.
    \end{outline}
\end{frame}

\begin{frame}[fragile]{Type}
    \begin{outline}
        \1<1-> Every value is an object, and every object has a \alert{type}.
        \1<2-> You can check the type of an object using the \texttt{type()} function.
        \begin{lstlisting}[language=Python]
type("Hello, World!")   # str
type(42)                # int
type(3.14)              # float
type(True)              # bool
type([1, 2, 3])         # list
type((1, 2, 3))         # tuple
type({"a": 1, "b": 2})  # dict
        \end{lstlisting}
        \1<3-> This is useful when you see \texttt{TypeError: ...} in your code, which means you are using a value of the wrong type.
    \end{outline}
\end{frame}

\begin{frame}{\texttt{pandas}}
    \begin{outline}
        \1<1-> \texttt{pandas} is a powerful package for data manipulation and analysis.
        \vspace{3ex}
        \1<2-> It introduces two classes that can hold any type.
            \2 \texttt{Series}, a one-dimensional labeled array (Column)
            \2 \texttt{DataFrame}, a two-dimensional labeled data structure (Table)
        \vspace{3ex}
        \1<3-> \texttt{pandas} is nothing but a combination of functions written by the contributors.
    \end{outline}
\end{frame}

\begin{frame}{}
    \begin{figure}
        \centering
        \includegraphics[width=\textwidth]{assets/pandas_github.png}
    \end{figure}
\end{frame}

\begin{frame}[fragile]{\texttt{pandas} (Cont.)}
    \begin{outline}
        \1<1-> To use any package, you need to import it first.
        \begin{lstlisting}[language=Python]
import pandas as pd
        \end{lstlisting}
        \1<2-> This line imports the \texttt{pandas} package and names it \texttt{pd} for short.
        \vspace{3ex}
        \1<3-> You can now use the functions in \texttt{pandas} by \texttt{pd.function\_name()}.
            \2 For example, \texttt{pd.DataFrame()} is a function that returns a new DataFrame object.
    \end{outline}
\end{frame}

\begin{frame}[fragile]{\texttt{pandas} (Cont.)}
    \begin{outline}
        \1<1-> An object also has \alert{attributes} (values) and \alert{methods} (functions).
        \1<2-> You can check them using the \texttt{dir()} function.
        \begin{lstlisting}[language=Python]
df = pd.DataFrame({
    'A': [1, 2, 3],
    'B': [4, 5, 6]
})
dir(df)
        \end{lstlisting}
        \1<3-> To access an attribute or method, use the dot notation on the object.
        \begin{lstlisting}[language=Python]
df.columns  # attribute
df.head()   # method
        \end{lstlisting}
    \end{outline}
\end{frame}

\begin{frame}{\texttt{pandas} (Cont.)}
    \begin{outline}
        \1<1-> Many packages including \texttt{pandas} have a gigantic number of functions. You don't need to know all of them.
        \1<2-> For all packages, you can learn via the following channels:
            \2 Official documentation (\url{https://pandas.pydata.org/docs/} for \texttt{pandas})
            \2 AI agents
                \3 You need to be extremely careful when using AI agents to learn packages. The core reason is that many AI agents are trained on the \alert{previous} version of a package, which can be very different from the current version.
                \3 For example, AI agents used to suggest \texttt{pd.DataFrame.append()} for adding a new row, but this method has become deprecated since version 1.4.0.
                \3 \texttt{pandas} just released version 3.0.0 on January 21, 2026, which has many breaking changes including introducing \texttt{pd.col()} syntax.
            \2 Online courses and tutorials
    \end{outline}
\end{frame}

\begin{frame}{\texttt{sklearn}}
    \begin{outline}
        \1<1-> \texttt{sklearn} is a powerful package for machine learning.
        \1<2-> To install \texttt{sklearn}, run \texttt{conda install scikit-learn}.
        \1<3-> It provides many functions for data preprocessing, model training, evaluation, etc.
        \vspace{3ex}
        \1<4-> For the first assignment, we will use the \texttt{CountVectorizer} (BoW), \texttt{TfidfVectorizer} (TF-IDF), and \texttt{KNeighborsClassifier} (KNN) functions in \texttt{sklearn}.
    \end{outline}
\end{frame}

\begin{frame}[fragile]{\texttt{CountVectorizer}}
    \begin{outline}
        \1<1-> Whenever you see a new function, you should first check the \alert{official documentation}. \url{https://scikit-learn.org/stable/modules/generated/sklearn.feature_extraction.text.CountVectorizer.html}
        \vspace{3ex}
        \1<2-> Some packages also have a more detailed \alert{user guide}. \url{https://scikit-learn.org/stable/modules/generated/sklearn.feature_extraction.text.CountVectorizer.html}
        \vspace{3ex}
        \1<3-> \texttt{CountVectorizer} is a class with the following parameters, attributes, and methods.
            \2 \texttt{from sklearn.feature\_extraction.text import CountVectorizer}
            \2 Parameters: \texttt{tokenizer}, \texttt{stop\_words}, \texttt{token\_pattern}
            \2 Methods: \texttt{fit}, \texttt{transform}, \texttt{fit\_transform}
    \end{outline}
\end{frame}

\begin{frame}{\texttt{tokenizer}}
    \begin{outline}
        \1<1-> The \texttt{tokenizer} parameter is a function that takes a string as input and returns a list of tokens (words).
        \1<2-> If you don't specify a \texttt{tokenizer}, \texttt{CountVectorizer} will use a default tokenizer that splits the string into words based on whitespace and punctuation.
            \2 For example, the string "Hello, World!" will be tokenized into ["Hello", "World"].
        \vspace{3ex}
        \1<3-> You can also specify a custom \texttt{tokenizer} if you want to use a different tokenization method.
            \2 For example, you can use \texttt{jieba} for Chinese tokenization. Then, the string \texttt{"你好,世界!"} will be tokenized into \texttt{["你好", ",", "世界", "!"]}.
    \end{outline}
\end{frame}

\begin{frame}{\texttt{stop\_words}}
    \begin{outline}
        \1<1-> The \texttt{stop\_words} parameter is a list of words that will be ignored during tokenization.
        \1<2-> If you set \texttt{stop\_words="english"}, \texttt{CountVectorizer} will use a built-in list of common English stop words (e.g., "the", "is", "in", etc.).
        \vspace{3ex}
        \1<3-> You can also specify your own \alert{list} of stop words if you want to ignore different words.
            \2 For example, you can use a custom list of stop words for Chinese, including \texttt{"综上所述", "总的来看", "总的来说"}. You may refer to \url{https://github.com/goto456/stopwords/tree/master}.
    \end{outline}
\end{frame}

\begin{frame}{\texttt{token\_pattern}}
    \begin{outline}
        \1<1-> The \texttt{token\_pattern} parameter is a regular expression that defines the pattern for tokenization.
        \1<2-> The default value is \texttt{r'(?u)\textbackslash b\textbackslash w\textbackslash w+\textbackslash b'}, which means that tokens must be at least 2 characters long and consist of word characters (letters, digits, or underscores).
        \vspace{3ex}
        \1<3-> You can change the regular expression to include different types of tokens.
            \2 For example, if you want to only include alphabetic tokens, you can set \texttt{token\_pattern=r'[a-zA-Z]+'}. Then, the string "Hello, World! 123" will be tokenized into ["Hello", "World"].
            \2 If you want to include Chinese characters, you can set \texttt{token\_pattern=r'[\textbackslash u4e00-\textbackslash u9fff]+'}. Then, the string \texttt{"你好,世界!"} will be tokenized into \texttt{["你好", "世界"]}.
    \end{outline}
\end{frame}

\begin{frame}{\texttt{fit}, \texttt{transform}, \texttt{fit\_transform}}
    \begin{outline}
        \1<1-> The \texttt{fit} method learns the vocabulary from the input data.
        \1<2-> The \texttt{transform} method transforms the input data into a document-term matrix based on the learned vocabulary.
        \1<3-> The \texttt{fit\_transform} method is a combination of \texttt{fit} and \texttt{transform}. It learns the vocabulary and transforms the input data in one step.
        \vspace{3ex}
        \1<4-> Since we train the model on the training data and apply the model to the test data, we should use \texttt{fit\_transform} on the training data and \texttt{transform} on the test data to avoid data leakage.
    \end{outline}
\end{frame}

\begin{frame}{Further Reading}
    \begin{outline}
        \1<1-> For more details on \texttt{TfidfVectorizer}, please refer to \url{https://scikit-learn.org/stable/modules/generated/sklearn.feature_extraction.text.TfidfVectorizer.html} and \url{https://scikit-learn.org/stable/modules/feature_extraction.html\#tfidf}.
        \vspace{3ex}
        \1<2-> For more details on \texttt{KNeighborsClassifier}, please refer \url{https://scikit-learn.org/stable/modules/generated/sklearn.neighbors.KNeighborsClassifier.html} and \url{https://scikit-learn.org/stable/modules/neighbors.html\#classification}.
    \end{outline}
\end{frame}

\begin{frame}{AI Agents}
    \begin{outline}
        \1<1-> Some of the students have already submitted the first assignment with the help of AI agents.
        \1<2-> My test on Github Copilot using the following prompt:
            \2 ``I am working in a uv project and have already run uv add datasets pandas scikit-learn. Based on the first question in \#file:assignment\_1.tex, generate Python code for a Jupyter Notebook. The code must load the ag\_news dataset from Hugging Face using the datasets library. Provide the solution logic for the question using pandas for data manipulation.''
    \end{outline}
\end{frame}

\end{document}